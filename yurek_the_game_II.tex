% % % % %
% Fiszki z teorii statystyki %
% by: Marcin Sidorowicz %
% % % % %

\documentclass[avery5371, grid, frame]{flashcards}

\usepackage[utf8]{inputenc}
\usepackage[T1]{fontenc}
\usepackage{amsmath}
\usepackage{amssymb}
\usepackage{textcomp}
\usepackage{mathtools}

\geometry{headheight=12pt}
\usepackage{fancyhdr}
\pagestyle{fancy}
\fancyhf{}
\renewcommand{\headrulewidth}{0pt}

\cardfrontstyle[]{headings}
\cardfrontfoot{Teoria prawdopodobieństwa 2}


\begin{document}

% % % % % % % % % % % %
% Rozkłady graniczne  %
% % % % % % % % % % % %

\begin{flashcard}[Definicja]{Rozkład (miara probabilistyczna) nieskończenie podzielny}
    Miarę probabilistyczną $\mu$ nazywamy nieskończenie podzielną, jeżeli 
    $$ \forall k \geq 2, \exists \nu \in \mathcal{P}(\mathbb{R}),  \nu^{*k} = \mu.$$
    Równoważnie:
    $\varphi \in ID$, gdy  $\forall k \geq 2, \exists \psi - f.char., (\psi(t))^k = \varphi(t).$
\end{flashcard}

\begin{flashcard}[Twierdzenie]{Funkcja charakterystyczna złożonego rozkładu Poissona}
    Jeżeli $X_1, X_2, \dots, X_n$ są i.i.d. z rozkładu $\mu$ i $Z_\lambda \sim Pois(\lambda)$ jest niezależna od $X_i$, to zm. los. $\sum_{i=1}^{Z_\lambda} X_i$ ma funkcję charakterstyczną $\varphi(t) = e^{\lambda (\varphi_\mu(t) - 1)}$.
\end{flashcard}

\begin{flashcard}[Twierdzenie]{Własności funkcji charakterystycznych ID}
    \begin{enumerate}
        \item Jeżeli $\varphi \in ID$, to $\varphi(t) \neq 0$ dla $t \in \mathbb{R}$,
        \item $(ID, \textasteriskcentered, \Rightarrow)$ jest domkniętą półgrupą,
        \item Jeżeli $X \in ID$, to $aX + b \in ID$ dla $a, b \in \mathbb{R}$.
    \end{enumerate}
\end{flashcard}

\begin{flashcard}[Twierdzenie]{Generowanie funkcji char. ID z miar borelowskich}
    Dla skończonej miary borelowskiej $m$ na $\mathbb{R}$ funkcja 
    $$ \varphi(t) = \exp \left (\int_\mathbb{R} (e^{itx} - 1 - itx) \frac{1}{x^2} dm \right) $$ jest funkcją charakterystyczną rozkładu ID. Uwaga: to twierdzenie nie opisuje wszystkich funkcji ID.
\end{flashcard}

\begin{flashcard}[Definicja]{Układ trójkątny}
    $X_{n, k}, k \in \{1, 2, \dots, k_n \}$ stoch. niezależne, $\mathbb{E}X_{n, k} = 0, Var(X_{n, k}) = \sigma^2_{n, k}$, $s_n^2 = \sum Var(X_{n, k})$. \\
    $\sup_n s_n^2 < \infty$, $\max_k \sigma_{n,k}^2 \xrightarrow{n} 0$.
\end{flashcard}

\begin{flashcard}[Twierdzenie]{Układy trójkątne a nieskończona podzielność}
    Jeśli $X \in ID_2^0$ (tzn. $X \in ID$ i $\mathbb{E}(X^2) < \infty$), to istnieje taki układ trójkątny, że $\sum_1^{k_n} X_{n, k} \xRightarrow[n]{} X$.
\end{flashcard}

\begin{flashcard}[Twierdzenie]{Wzór Kołmogorowa}
    Jeśli w układzie trójkątnym mamy $\sum_1^{k_n} \xRightarrow[n]{} X$, to $X \in ID_2$ oraz istnieje $m \in \mathcal{P}(\mathbb{R})$ t. że
    $$ \varphi(t) = \exp \left (\int_\mathbb{R} (e^{itx} - 1 - itx) \frac{1}{x^2} dm \right), $$ gdzie $m(\mathbb{R}) = \mathbb{E}(X^2)$.
\end{flashcard}

\begin{flashcard}[Twierdzenie]{Wzór L\'evy'ego-Chinczyna}
\begin{footnotesize}
    $X \in ID \iff \exists a \in \mathbb{R}, \sigma \geq 0, \exists G \in  \mathcal{M}(\mathbb{R})$, t.że
    $$ \varphi_X(t) = \exp \left (ita - \frac{1}{2}t^2\sigma^2 + \int_\mathbb{R} (e^{itx} - 1 - \frac{itx}{1+x^2}) \frac{itx^2}{x^2} dG
    \right)$$
    Można zapisać inaczej:
    $$ \varphi_X(t) = \exp \left (ita - \frac{1}{2}t^2\sigma^2 + \int_{\mathbb{R} \textbackslash \{ 0 \}} (e^{itx} - 1 - \frac{itx}{1+x^2}) dM
    \right)$$
    gdzie $M$ - miara skończona na $\mathbb{R} \textbackslash \{0\}$, $M( \{ x : |x| > \epsilon\} < \infty$, $\int_{\mathbb{R}} 1 \wedge x^2 dM < \infty$. $a$, $\sigma$, $M$ są jednoznaczne.
\end{footnotesize}
\end{flashcard}

\begin{flashcard}[Definicja]{Rozkład samorozkładalny}
    Niech $(Z_n)$ będą stoch. niezależne, $a_n > 0$, $b_n \in \mathbb{R}$ takie że
    \begin{enumerate}
        \item Układ $\{a_n Z_k : 1 \leq k \leq n, n \geq 1 \}$ jest infinitezymalny,
        \item $a_n(Z_1 + Z_2 + \dots + Z_n) \Rightarrow \mu$.
    \end{enumerate}
    Wtedy $\mu$ nazywamy samorozkładalną (L - klasa rozkładów samorozkładalnych).
\end{flashcard}

\begin{flashcard}[Twierdzenie]{Twierdzenie Chinczyna o zbieżności typów}
    Jeżeli $X_n \Rightarrow X$, $Y_n := a_n X_n \Rightarrow Y$ dla $a_n > 0$, $b_n \in \mathbb{R}$ oraz $X$ i $Y$ są niezdegenerowane, to $a_n \rightarrow a$ i $b_n \rightarrow b$, gdzie $Y = aX + b$.
\end{flashcard}

\begin{flashcard}[Twierdzenie]{Twierdzenie L\'evy'ego (i wniosek)}
    $X \in L$ wtedy i tylko wtedy, gdy $\forall c \in (0, 1), \exists X_c \bot X, X \overset{d}{=} cX + X_c$. \\
    Równoważnie, gdy $\varphi_X(t) = \varphi_X(ct) + \psi_c(t)$. \\
    Wniosek: $X \sim [a, \sigma^2, M] \in L \iff \forall c \in (0, 1), \forall B \subseteq \mathbb{R} \textbackslash \{0\}, M(B) \geq T_c M(B) = M(c^{-1} B)$.
\end{flashcard}

\begin{flashcard}[Twierdzenie]{Twierdzenie Urbanika}
\begin{small}
    $\varphi : \mathbb{R} \rightarrow \mathbb{C}$ jest funkcją charakterystyczną rozkładu z klasy L wtedy i tylko wtedy, gdy
    \begin{multline*}
        \varphi_X(t) = \exp (itx_0 - \frac{1}{4}t^2\sigma^2 + \\ \int_{\mathbb{R} \textbackslash \{ 0 \}} (\int_0^1 \frac{e^{itxw}}{w}dw - 1 - itx \chi_{|x|<1}) dM)
    \end{multline*}
    gdzie $M$ jest miarą L\'evy'ego oraz $\int_{|x| < \epsilon} \log(1+|x|)dM < \infty$. Ponadto trójka $x_0, \sigma^2, M$ jest jednoznaczna.
\end{small}
\end{flashcard}

\begin{flashcard}[Twierdzenie]{Twierdzenie Verwaata}
    $$\varphi_x \in L
    \iff X \overset{d}{=} \int_0^\infty e^{-s} dY(s),$$
    gdzie $Y(s)$ jest procesem L\'evy'ego.
\end{flashcard}

\begin{flashcard}[Definicja]{Rozkład stabilny}
    Zmienna losowa $Y$ jest stabilna, gdy istnieje $a_n > 0$, ciąg $(V_n)$ i.i.d. oraz $b_n \in \mathbb{R}$ takie, że $a_n(V_1 + \dots + V_n) + b_n \Rightarrow Y$.
\end{flashcard}

\begin{flashcard}[Twierdznie]{Równanie splotowe rozkładu stabilnego}
    NWSR:
    \begin{enumerate}
        \item $\mu$ jest stabilna,
        \item $\exists \alpha > 0, \forall n \geq 1, \exists Z_n \in \mathbb{R}, \mu^{\textasteriskcentered n} = T_{n^{1/\alpha}} \mu \textasteriskcentered \delta_{Z_n}$,
        \item $\exists \alpha > 0, \forall t > 0, \exists Z_t \in \mathbb{R}, \mu^{\textasteriskcentered n} = T_{t^{1/\alpha}} \mu \textasteriskcentered \delta_{Z_t}$
    \end{enumerate}
\end{flashcard}

\begin{flashcard}[Twierdzenie]{Równanie splotowe sumy rozkładów stabilnych i ograniczenie wykładnika $\alpha$}
    \begin{itemize}
        \item $\exists \alpha > 0, \forall a, b > 0, \exists Z(a, b) \in \mathbb{R}, T_a \mu \textasteriskcentered T_b \mu = T_{(a^\alpha + b^\alpha)^{1/\alpha}} \mu \textasteriskcentered \delta_{Z_(a, b)}$
        \item Jeśli niezdegenerowana $\mu$ jest stabilna, to $\alpha \in (0, 2]$. Ponadto $\mu$ jest albo gaussowska ($\alpha = 2$), albo niesk. podzielna bez części gaussowskiej.
    \end{itemize}
\end{flashcard}

\begin{flashcard}[Twierdzenie]{Postać symetrycznej, nie-gaussowskiej zmiennej stabilnej}
$$ \varphi_X (t)= \exp (-c_\alpha |t|^\alpha) =
\exp \left( \int_0^\infty (\cos tx - 1) \frac{dx}{x^{1+\alpha}} \right)$$
\end{flashcard}

% % % % % % % % % % % % % % % % %
% Warunkowa wartość oczekiwana  %
% % % % % % % % % % % % % % % % %

\begin{flashcard}[Definicja]{Warunkowa wartość oczekiwana}

\smallskip
Niech $ X \in L_1 (P)$, $ \mathcal{G} \subseteq  \mathcal{F}$ \\
Wtedy $\mathbb{E} [ X | \mathcal{G} ]$ to funkcja $g$ t. że:
\begin{itemize}
\item $g$ jest $\mathcal{G}$-mierzalna
\item $\forall G \in \mathcal{G} \int_G X \, dP = \int_G g \, dP$
\end{itemize}
Ponadto $g$ jest jedyną taką funkcją z $P1$. \\
Jeśli $X \in L_2(P)$, to $\mathbb{E} [ X | \mathcal{G} ]$ jest rzutem na $L_2(\mathcal{G})$.
\end{flashcard}

\begin{flashcard}[Twierdzenie]{Własności warunkowej wartości oczekiwanej}

Dla $X, Y \in L_1(P)$, $a, b, c \in \mathbb{R}$:
\begin{itemize}
\item $\mathbb{E}[aX +bY | \mathcal{G}] = a \, \mathbb{E}[ X | \mathcal{G} ] +
                                          b \, \mathbb{E}[ Y | \mathcal{G} ]$ z $P1$.
\item $\mathbb{E}[c | \mathcal{G} ] = c $ z $P1$.
\item Jeśli $X \leq Y $ z $P1$, to $ \mathbb{E}[X | \mathcal{G}] \leq \mathbb{E}[Y | \mathcal{G}]$ z $P1$.
\item $\mathbb{E} [ \mathbb{E} [ X | \mathcal{G} ] ] = \mathbb{E}[X]$.
\item Jeśli $XY \in L_1(P)$, $Y$ : $\mathcal{G}$-mierzalne, to $\mathbb{E}[XY|\mathcal{G}]] =
                                                                Y\mathbb{E}[X|\mathcal{G}]]$ z $P1$.
\end{itemize}
\end{flashcard}

\begin{flashcard}[Twierdzenie]{Mniejsza $\sigma$-algebra wygrywa}

\smallskip
Niech $\mathcal{G}_1 \subseteq \mathcal{G}_2$: $\sigma$-algebry. \\
Wtedy $\mathbb{E} [ \mathbb{E} [ X | \mathcal{G}_2] | \mathcal{G}_1 ] =
       \mathbb{E} [ \mathbb{E} [ X | \mathcal{G}_1] | \mathcal{G}_2 ] =
       \mathbb{E} [ X | \mathcal{G}_1]$
\end{flashcard}

% % % % % % % %
% Martyngały  %
% % % % % % % %

\begin{flashcard}[Definicja]{Martyngał}
\smallskip
Ustalmy $ \left( \Omega, \mathcal{F}, \left( \mathcal{F}_i \right) _{i < \omega}, P \right) $ t. że $\mathcal{F}_0 \subseteq \mathcal{F}_1 \subseteq ... \subseteq \mathcal{F}$. \\
Jeśli ciąg $\left( X_n \right) _{n < \omega} $ spełnia:
\begin{itemize}
\item $ X_n \in L_1(P) $
\item $ X_n $: $ \mathcal{F}_n $-mierzalny
\item $ \mathbb{E} [ X_{n+1} | \mathcal{F}_n ] = X_n $ z $P1$
\end{itemize}
to jest martyngałem. ($ \leq $: nadmartyngałem [supermartyngałem], $ \geq $: submartyngałem)
\end{flashcard}

\begin{flashcard}[Definicja]{Ciąg prognozowalny}

\smallskip
Ciąg $ \left( H_n \right) _{n < \omega} $ jest prognozowalny, jeśli $H_{n+1}$ jest $\mathcal{F}_n$-mierzalny.
\end{flashcard}

\begin{flashcard}[Definicja]{Czas zatrzymania}

\smallskip
$\tau: \Omega \rightarrow \{1, 2, ...\} \cup \{ \infty \}$ jest czasem zatrzymania, gdy $(\forall k \in \mathbb{N}) \, \{ \tau \leq k \} \in \mathcal{F}_k$ \\
Jeżeli $\tau$ jest czasem zatrzymania, to $\mathbf{1}_{[\tau \geq n]}$ jest prognozowalny.
\end{flashcard}

\begin{flashcard}[Twierdzenie]{Twierdzenie A \\ Transformata martyngałowa}

\smallskip
Jeśli $(X_n)$: martyngał, $(H_n)$: prognozowalny, oraz $H_n$ są ograniczone z $P1$, to
$$ (H \circ X)_n = \sum ^n _j H_j (X_j - X_{j-1})  $$
jest martyngałem. \\
Powyższe twierdzenie jest prawdziwe dla submartyngałów przy dodatkowym założeniu nieujemności $H_n$.
\end{flashcard}

\begin{flashcard}[Definicja]{Funkcja wypukła (a martyngały)}

\smallskip
Funkcja $\varphi$ jest wypukła, gdy $\varphi(x) = \sup \{ ax+b : (a, b) \in S\}$, gdzie $S := \{(a, b) : a, b \in \mathbb{Q} \; \text{i} \; (\forall x) \, ax+b \leq \varphi(x) \}$. \\
Jeśli $(X_n)$: mtg i $\varphi$: wypukła, to $(\varphi(X_n))$ jest submtg. \\
Jeśli $(X_n)$: submtg i $\varphi$: wypukła, niemalejąca, to $(\varphi(X_n))$ jest submtg.
\end{flashcard}

\begin{flashcard}[Twierdzenie]{Twierdzenie B \\ Upcrossing inequality}

\smallskip
Dla submtg $(X_n)$, $ a < b $
$$ (b-a) \mathbb{E}[U_n] \leq \mathbb{E}[(X_n-a)^+] - \mathbb{E}[(X_0 - a)^+]$$
\end{flashcard}

\begin{flashcard}[Twierdzenie]{Twierdzenie C \\ Zbieżność submartyngałów}

\smallskip
Jeśli $(X_n)$: submtg, $\sup _n \mathbb{E}[X ^+ _n] < K < \infty$, to istnieje $X_\infty \in L_1$ t. że $X_n \rightarrow X_\infty$ z $P1$.
\end{flashcard}

\begin{flashcard}[Twierdzenie]{Rozkład Dooba}

\smallskip
Każdy submtg $(X_n)$ ma jednoznaczny rozkład $X_n = M_n + A_n$ gdzie $(M_n)$: mtg, $A_0 = 0$, $A_n \geq 0$: niemalejący, prognozowalny.
\end{flashcard}

\begin{flashcard}[Twierdzenie]{Nierówność Dooba dla submartyngałów}

\smallskip
Jeśli $(X_n)$: submartyngał i $\lambda > 0$, to:
$$ \lambda \, \text{P} \left[ \max _{ 0 \leq j \leq m} X_j \geq \lambda \right] \leq \mathbb{E}[X ^+ _m] $$
\end{flashcard}

% % % % % % % % %
% Ruchy Browna  %
% % % % % % % % %

\begin{flashcard}[Definicja]{Ruch Browna \\ (Proces Wienera)}

Rodzina zmiennych losowych $ \left( W_t \right) _{t > 0} $ t. że:
\begin{itemize}
\item $W_0 \equiv 0$ z $P1$
\item $W_t - W_s \stackrel{d}{=} N(0, t-s)$ dla $ t > s $
\item $(\forall {n \geq 2}) \, (\forall { 0 \leq t_1 \leq ... \leq t_n }) \, W_{t_j} - W_{t_i}$ są s.n.
\item $t \mapsto W_t(\omega)$ są ciągłe z $P1$
\end{itemize}
jest nazywana ruchem Browna bądź procesem Wienera. Będziemy też używać oznaczeń $W(t)$, $W(t; \omega)$.
\end{flashcard}

\begin{flashcard}[Twierdzenie]{Równoważna definicja ruchu Browna}

\smallskip
Rodzina zmiennych losowych $ \left( W_t \right) _{t > 0} $ t. że:
\begin{itemize}
\item $W(0) = 0$ z $P1$
\item rozkłady skończone są normalne w $\mathbb{R}^n$
\item $\text{Cov}(W(t), W(s)) = t \wedge s$
\item $t \mapsto W(t; \omega)$ jest ciągła z $P1$
\end{itemize}
jest ruchem Browna.
\end{flashcard}

\begin{flashcard}[Twierdzenie]{Różniczkowalność procesu Wienera}

\smallskip
(Wiener, 1930) Trajektorie procesu Wienera są nigdzie nieróżniczkowalne z $P1$.
\end{flashcard}

\begin{flashcard}[Twierdzenie]{Przekształcanie ruchu Browna}

\smallskip
Jeśli $W(t)$, $t \geq 0$ jest procesem Wienera, to:
\begin{itemize}
\item $W_1(t) := -W(T)$
\item $W_2(t) := W(t+c) - W(c)$
\item $W_3(t) := tW(\frac{1}{t}), W_3(0)=0$ 
\end{itemize}
też są.	
\end{flashcard}

\begin{flashcard}[Twierdzenie]{Mocna własność Markowa}

\smallskip
Niech $W_t$: proces Wienera, $\mathcal{F}_t := \sigma (W_s : s \leq t) $ \\
Ponadto $\tau: \Omega \rightarrow [0, \infty]$ jest czasem Markowa, tj. $$ \{ \tau \leq t \} \in \mathcal{F}_t$$
Wtedy $W ^* _t := W(t + \tau(\omega), \omega) - W(\tau(\omega), \omega)$ jest procesem Wienera niezależnym od $\mathcal{F}_\tau := \{ A \in \mathcal{F} : \forall t \geq 0 \; ( \{\tau \leq t \} \cap A) \in \mathcal{F}_t \} $ ($\sigma$-algebra zatrzymana).
\end{flashcard}

\begin{flashcard}[Twierdzenie]{Zasada odbicia}

\smallskip
$$\text{P} [ \sup _{ s \leq t } W_s > a ] = 2 \text{P} [ W_t > a ] $$ 
\end{flashcard}

\begin{flashcard}[Twierdzenie]{Prawo iterowanego logarytmu}

\smallskip
$$ \text{P} \left[ \limsup _{t \rightarrow 0} \frac{W(t)}{\sqrt{2t \log \log \frac{1}{t} }} \right] = 1 $$
\end{flashcard}

% % % % % % % % % % % % % % % %
% Martyngały z czasem ciągłym %
% % % % % % % % % % % % % % % %

\begin{flashcard}[Definicja]{Martyngał z czasem ciągłym}

\smallskip
Martyngał z czasem ciągłym to proces stochastyczny $(M_t)_{t \geq 0}$ spełniający:
\begin{itemize}
\item $\mathbb{E} [|M_t|] < \infty$
\item $M_t$: $\mathcal{F}_t$-mierzalna
\item $\forall t > s \; \mathbb{E} \, [ M_t | \mathcal{F}_s ] = M_s$ z $P1$
\end{itemize}
\end{flashcard}

\begin{flashcard}[Twierdzenie]{Nierówność Dooba dla martyngałów z czasem ciągłym}

\smallskip
Niech $(M_t)_{t \geq 0}$: martyngał o ciągłych trajektoriach. Wtedy:
$$ \forall \lambda > 0 \; \text{P} \left[ \sup _{s \leq t} M_t \geq \lambda \right] \leq \lambda \mathbb{E} \, [M ^+ _t ] $$
gdzie $ M ^+ _t = M_t \lor 0 ]$.
\end{flashcard}

% % % % % % %
% Michaszki %
% % % % % % %

\begin{flashcard}[Definicja]{Proces L\'evy'ego}

\smallskip
Proces Lévy'eqo to rodzina zmiennych losowych $(X_t)_{t \geq 0}$ t. że:
\begin{itemize}
\item $X_0 \equiv 0$ z $P1$
\item $\forall t > s \; X_t - X_s \stackrel{d}{=} X_{t-s} - X_0 $
\item przyrosty są niezależne
\item $ t \mapsto X_t $ ciągła wg. P, tj. $\text{P} [ | X_t -X_s | > \varepsilon ] \stackrel{s \rightarrow t}{\rightarrow} 0 $
\end{itemize}
\end{flashcard}

\begin{flashcard}[Twierdzenie]{Tożsamość Walda}

\smallskip
$$ \mathbb{E} \left[ \sum _{n = 1} ^{n = \tau} X_n \right] = \mathbb{E}\left[\tau \right] \mathbb{E}\left[X_1 \right] $$
\end{flashcard}

\end{document}