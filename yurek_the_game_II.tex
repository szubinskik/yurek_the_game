\documentclass[avery5371,grid,frame]{flashcards}

\usepackage{polski}
\usepackage[utf8]{inputenc}
\usepackage{amsmath,amssymb,mathtools}


\cardfrontstyle[\large\slshape]{headings}
\cardbackstyle{empty}


\begin{document}

\cardfrontfoot{Teoria Prawdopodobieństwa}

% % % % % % % % % % % % % % % % %
% Warunkowa wartość oczekiwana  %
% % % % % % % % % % % % % % % % %

\begin{flashcard}[Definicja]{Warunkowa wartość oczekiwana}

\smallskip
Niech $ X \in L_1 (P)$, $ \mathcal{G} \subseteq  \mathcal{F}$ \\
Wtedy $\mathbb{E} [ X | \mathcal{G} ]$ to funkcja $g$ t. że:
\begin{itemize}
\item $g$ jest $\mathcal{G}$-mierzalna
\item $\forall G \in \mathcal{G} \int_G X \, dP = \int_G g \, dP$
\end{itemize}
Ponadto $g$ jest jedyną taką funkcją z $P1$. \\
Jeśli $X \in L_2(P)$, to $\mathbb{E} [ X | \mathcal{G} ]$ jest rzutem na $L_2(\mathcal{G})$.
\end{flashcard}

\begin{flashcard}[Twierdzenie]{Własności warunkowej wartości oczekiwanej}

Dla $X, Y \in L_1(P)$, $a, b, c \in \mathbb{R}$:
\begin{itemize}
\item $\mathbb{E}[aX +bY | \mathcal{G}] = a \, \mathbb{E}[ X | \mathcal{G} ] +
                                          b \, \mathbb{E}[ Y | \mathcal{G} ]$ z $P1$.
\item $\mathbb{E}[c | \mathcal{G} ] = c $ z $P1$.
\item Jeśli $X \leq Y $ z $P1$, to $ \mathbb{E}[X | \mathcal{G}] \leq \mathbb{E}[Y | \mathcal{G}]$ z $P1$.
\item $\mathbb{E} [ \mathbb{E} [ X | \mathcal{G} ] ] = \mathbb{E}[X]$.
\item Jeśli $XY \in L_1(P)$, $Y$ : $\mathcal{G}$-mierzalne, to $\mathbb{E}[XY|\mathcal{G}]] =
                                                                Y\mathbb{E}[X|\mathcal{G}]]$ z $P1$.
\end{itemize}
\end{flashcard}

\begin{flashcard}[Twierdzenie]{Mniejsza $\sigma$-algebra wygrywa}

\smallskip
Niech $\mathcal{G}_1 \subseteq \mathcal{G}_2$: $\sigma$-algebry. \\
Wtedy $\mathbb{E} [ \mathbb{E} [ X | \mathcal{G}_2] | \mathcal{G}_1 ] =
       \mathbb{E} [ \mathbb{E} [ X | \mathcal{G}_1] | \mathcal{G}_2 ] =
       \mathbb{E} [ X | \mathcal{G}_1]$
\end{flashcard}

\begin{flashcard}[]{}
\end{flashcard}

% % % % % % % %
% Martyngały  %
% % % % % % % %

\begin{flashcard}[Definicja]{Martyngał}
\smallskip
Ustalmy $ \left( \Omega, \mathcal{F}, \left( \mathcal{F}_i \right) _{i < \omega}, P \right) $ t. że $\mathcal{F}_0 \subseteq \mathcal{F}_1 \subseteq ... \subseteq \mathcal{F}$. \\
Jeśli ciąg $\left( X_n \right) _{n < \omega} $ spełnia:
\begin{itemize}
\item $ X_n \in L_1(P) $
\item $ X_n $: $ \mathcal{F}_n $-mierzalny
\item $ \mathbb{E} [ X_{n+1} | \mathcal{F}_n ] = X_n $ z $P1$
\end{itemize}
to jest martyngałem. ($ \leq $: nadmartyngałem [supermartyngałem], $ \geq $: submartyngałem)
\end{flashcard}

\begin{flashcard}[Definicja]{Ciąg prognozowalny}

\smallskip
Ciąg $ \left( H_n \right) _{n < \omega} $ jest prognozowalny, jeśli $H_{n+1}$ jest $\mathcal{F}_n$-mierzalny.
\end{flashcard}

\begin{flashcard}[]{}
\end{flashcard}

% % % % % % % % %
% Ruchy Browna  %
% % % % % % % % %

\begin{flashcard}[Definicja]{Ruch Browna \\ (Proces Wienera)}

Rodzina zmiennych losowych $ \left( W_t \right) _{t > 0} $ t. że:
\begin{itemize}
\item $W_0 \equiv 0$ z $P1$
\item $W_t - W_s \stackrel{d}{=} N(0, t-s)$ dla $ t > s $
\item $(\forall {n \geq 2}) \, (\forall { 0 \leq t_1 \leq ... \leq t_n }) \, W_{t_j} - W_{t_i}$ są s.n.
\item $t \mapsto W_t(\omega)$ są ciągłe z $P1$
\end{itemize}
jest nazywana ruchem Browna bądź procesem Wienera. Będziemy też używać oznaczeń $W(t)$, $W(t; \omega)$.
\end{flashcard}

\begin{flashcard}[Twierdzenie]{Równoważna definicja ruchu Browna}

\smallskip
Rodzina zmiennych losowych $ \left( W_t \right) _{t > 0} $ t. że:
\begin{itemize}
\item $W(0) = 0$ z $P1$
\item rozkłady skończone są normalne w $\mathbb{R}^n$
\item $\text{Cov}(W(t), W(s)) = t \wedge s$
\item $t \mapsto W(t; \omega)$ jest ciągła z $P1$
\end{itemize}
jest ruchem Browna.
\end{flashcard}

\begin{flashcard}[Twierdzenie]{Różniczkowalność procesu Wienera}

\smallskip
(Wiener, 1930) Trajektorie procesu Wienera są nigdzie nieróżniczkowalne z $P1$.
\end{flashcard}

\begin{flashcard}[Twierdzenie]{Przekształcanie ruchu Browna}

\smallskip
Jeśli $W(t)$, $t \geq 0$ jest procesem Wienera, to:
\begin{itemize}
\item $W_1(t) := -W(T)$
\item $W_2(t) := W(t+c) - W(c)$
\item $W_3(t) := tW(\frac{1}{t}), W_3(0)=0$ 
\end{itemize}
też są.
\end{flashcard}

\begin{flashcard}[]{}
\end{flashcard}

\end{document}
