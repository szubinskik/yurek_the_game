\documentclass[avery5371,grid,frame]{flashcards}

\usepackage{polski}
\usepackage[utf8]{inputenc}
\usepackage{amsmath,amssymb}


\cardfrontstyle[\large\slshape]{headings}
\cardbackstyle{empty}


\begin{document}

\cardfrontfoot{Teoria Prawdopodobieństwa}


\begin{flashcard}[Twierdzenie]{Lematy Borela-Cantellego}

\smallskip
Niech $\{A_n\}_{n < \omega}$: ciąg zdarzeń losowych.
\begin{description}
\item[I lemat Borela-Cantellego] \hfill \\
	Jeśli $\sum_{n=0}^{\infty} P(A_n) < \infty$, to $P\left(\limsup\limits_{n \rightarrow \infty}{A_n}\right)=0$
\item[II lemat Borela-Cantellego] \hfill \\
	Jeśli $\sum_{n=0}^{\infty} P(A_n) = \infty$ i $A_n$ są stochastycznie niezależne, to $P\left(\limsup\limits_{n \rightarrow \infty}{A_n}\right)=1$ 
\end{description}
\end{flashcard}

\begin{flashcard}[Twierdzenie]{Twierdzenie Riesza}

\smallskip
Niech $X_n$: ciąg zmiennych losowych. \medskip \\ 
Jeśli $X_n \xrightarrow{n \rightarrow \infty} X$ wg. $P$, to istnieją $(n_i)_{i < \omega}$ takie, \\
że $X_{n_k} \xrightarrow{k \rightarrow \infty} X$ z $P1$.
\end{flashcard}

\begin{flashcard}[Definicja]{Ciasny (jędrny, tight) ciąg miar}

\smallskip
Ciąg miar jest ciasny, gdy
$$ (\forall \ \varepsilon > 0)(\exists \ a, b \in \mathbb{R})(\forall \ n > 0) \Big[\mu_n \left( (a,b] \right) > 1-\varepsilon \Big] $$
\end{flashcard}

\begin{flashcard}[Definicja]{$\pi$ i $\lambda$ systemy}

\smallskip
\begin{itemize}
\item Rodzina zbiorów $\mathit{P}$ jest $\pi$-systemem, gdy jest \\
zamknięta na przekroje
\item Rodzina zbiorów $\mathit{L}$ jest $\lambda$-systemem, gdy:
	\begin{enumerate}
	\item $\Omega \in \mathit{L}$
	\item $A \subseteq B$ i $A, B \in \mathit{L} \Rightarrow B \setminus A \in \mathit{L}$
	\item $A_0 \subseteq A_1 \subseteq \ldots \in \mathit{L} \Rightarrow \cup^{\infty}_{i=0} \ A_i \in \mathit{L}$
	\end{enumerate}

\end{itemize}
\end{flashcard}


\begin{flashcard}[Twierdzenie]{(blank)}
(blank)
\end{flashcard}
\begin{flashcard}[Twierdzenie]{(blank)}
(blank)
\end{flashcard}
\begin{flashcard}[Twierdzenie]{(blank)}
(blank)
\end{flashcard}
\begin{flashcard}[Twierdzenie]{(blank)}
(blank)
\end{flashcard}
\begin{flashcard}[Twierdzenie]{(blank)}
(blank)
\end{flashcard}
\begin{flashcard}[Twierdzenie]{(blank)}
(blank)
\end{flashcard}

\end{document}
