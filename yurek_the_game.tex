\documentclass[avery5371,grid,frame]{flashcards}

\usepackage{polski}
\usepackage[utf8]{inputenc}
\usepackage{amsmath,amssymb}


\cardfrontstyle[\large\slshape]{headings}
\cardbackstyle{empty}


\begin{document}

\cardfrontfoot{Teoria Prawdopodobieństwa}


\begin{flashcard}[Twierdzenie]{Lematy Borela-Cantellego}

Niech $\{A_n\}_{n < \omega}$: ciąg zdarzeń losowych.
\begin{description}
\item[I lemat Borela-Cantellego] \hfill \\
	Jeśli $\sum_{n=0}^{\infty} P(A_n) < \infty$, to $P\left(\limsup\limits_{n \rightarrow \infty}{A_n}\right)=0$
\item[II lemat Borela-Cantellego] \hfill \\
	Jeśli $\sum_{n=0}^{\infty} P(A_n) = \infty$ i $A_n$ są stochastycznie niezależne, to $P\left(\limsup\limits_{n \rightarrow \infty}{A_n}\right)=1$ 
\end{description}
\end{flashcard}

\begin{flashcard}[Twierdzenie]{Twierdzenie Riesza}

Niech $X_n$: ciąg zmiennych losowych. \medskip \\ 
Jeśli $X_n \xrightarrow{n \rightarrow \infty} X$ wg. $P$, to istnieją $(n_i)_{i < \omega}$ takie, \\
że $X_{n_k} \xrightarrow{k \rightarrow \infty} X$ z $P1$.
\end{flashcard}
\begin{flashcard}[Twierdzenie]{(blank)}
(blank)
\end{flashcard}
\begin{flashcard}[Twierdzenie]{(blank)}
(blank)
\end{flashcard}
\begin{flashcard}[Twierdzenie]{(blank)}
(blank)
\end{flashcard}
\begin{flashcard}[Twierdzenie]{(blank)}
(blank)
\end{flashcard}
\begin{flashcard}[Twierdzenie]{(blank)}
(blank)
\end{flashcard}
\begin{flashcard}[Twierdzenie]{(blank)}
(blank)
\end{flashcard}
\begin{flashcard}[Twierdzenie]{(blank)}
(blank)
\end{flashcard}
\begin{flashcard}[Twierdzenie]{(blank)}
(blank)
\end{flashcard}

\end{document}
